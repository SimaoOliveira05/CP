\documentclass[11pt, a4paper, fleqn]{article}
\usepackage{cp2526t}
\makeindex

%================= lhs2tex=====================================================%
%% ODER: format ==         = "\mathrel{==}"
%% ODER: format /=         = "\neq "
%
%
\makeatletter
\@ifundefined{lhs2tex.lhs2tex.sty.read}%
  {\@namedef{lhs2tex.lhs2tex.sty.read}{}%
   \newcommand\SkipToFmtEnd{}%
   \newcommand\EndFmtInput{}%
   \long\def\SkipToFmtEnd#1\EndFmtInput{}%
  }\SkipToFmtEnd

\newcommand\ReadOnlyOnce[1]{\@ifundefined{#1}{\@namedef{#1}{}}\SkipToFmtEnd}
\usepackage{amstext}
\usepackage{amssymb}
\usepackage{stmaryrd}
\DeclareFontFamily{OT1}{cmtex}{}
\DeclareFontShape{OT1}{cmtex}{m}{n}
  {<5><6><7><8>cmtex8
   <9>cmtex9
   <10><10.95><12><14.4><17.28><20.74><24.88>cmtex10}{}
\DeclareFontShape{OT1}{cmtex}{m}{it}
  {<-> ssub * cmtt/m/it}{}
\newcommand{\texfamily}{\fontfamily{cmtex}\selectfont}
\DeclareFontShape{OT1}{cmtt}{bx}{n}
  {<5><6><7><8>cmtt8
   <9>cmbtt9
   <10><10.95><12><14.4><17.28><20.74><24.88>cmbtt10}{}
\DeclareFontShape{OT1}{cmtex}{bx}{n}
  {<-> ssub * cmtt/bx/n}{}
\newcommand{\tex}[1]{\text{\texfamily#1}}	% NEU

\newcommand{\Sp}{\hskip.33334em\relax}


\newcommand{\Conid}[1]{\mathit{#1}}
\newcommand{\Varid}[1]{\mathit{#1}}
\newcommand{\anonymous}{\kern0.06em \vbox{\hrule\@width.5em}}
\newcommand{\plus}{\mathbin{+\!\!\!+}}
\newcommand{\bind}{\mathbin{>\!\!\!>\mkern-6.7mu=}}
\newcommand{\rbind}{\mathbin{=\mkern-6.7mu<\!\!\!<}}% suggested by Neil Mitchell
\newcommand{\sequ}{\mathbin{>\!\!\!>}}
\renewcommand{\leq}{\leqslant}
\renewcommand{\geq}{\geqslant}
\usepackage{polytable}

%mathindent has to be defined
\@ifundefined{mathindent}%
  {\newdimen\mathindent\mathindent\leftmargini}%
  {}%

\def\resethooks{%
  \global\let\SaveRestoreHook\empty
  \global\let\ColumnHook\empty}
\newcommand*{\savecolumns}[1][default]%
  {\g@addto@macro\SaveRestoreHook{\savecolumns[#1]}}
\newcommand*{\restorecolumns}[1][default]%
  {\g@addto@macro\SaveRestoreHook{\restorecolumns[#1]}}
\newcommand*{\aligncolumn}[2]%
  {\g@addto@macro\ColumnHook{\column{#1}{#2}}}

\resethooks

\newcommand{\onelinecommentchars}{\quad-{}- }
\newcommand{\commentbeginchars}{\enskip\{-}
\newcommand{\commentendchars}{-\}\enskip}

\newcommand{\visiblecomments}{%
  \let\onelinecomment=\onelinecommentchars
  \let\commentbegin=\commentbeginchars
  \let\commentend=\commentendchars}

\newcommand{\invisiblecomments}{%
  \let\onelinecomment=\empty
  \let\commentbegin=\empty
  \let\commentend=\empty}

\visiblecomments

\newlength{\blanklineskip}
\setlength{\blanklineskip}{0.66084ex}

\newcommand{\hsindent}[1]{\quad}% default is fixed indentation
\let\hspre\empty
\let\hspost\empty
\newcommand{\NB}{\textbf{NB}}
\newcommand{\Todo}[1]{$\langle$\textbf{To do:}~#1$\rangle$}

\EndFmtInput
\makeatother
%
%
%
%
%
%
% This package provides two environments suitable to take the place
% of hscode, called "plainhscode" and "arrayhscode". 
%
% The plain environment surrounds each code block by vertical space,
% and it uses \abovedisplayskip and \belowdisplayskip to get spacing
% similar to formulas. Note that if these dimensions are changed,
% the spacing around displayed math formulas changes as well.
% All code is indented using \leftskip.
%
% Changed 19.08.2004 to reflect changes in colorcode. Should work with
% CodeGroup.sty.
%
\ReadOnlyOnce{polycode.fmt}%
\makeatletter

\newcommand{\hsnewpar}[1]%
  {{\parskip=0pt\parindent=0pt\par\vskip #1\noindent}}

% can be used, for instance, to redefine the code size, by setting the
% command to \small or something alike
\newcommand{\hscodestyle}{}

% The command \sethscode can be used to switch the code formatting
% behaviour by mapping the hscode environment in the subst directive
% to a new LaTeX environment.

\newcommand{\sethscode}[1]%
  {\expandafter\let\expandafter\hscode\csname #1\endcsname
   \expandafter\let\expandafter\endhscode\csname end#1\endcsname}

% "compatibility" mode restores the non-polycode.fmt layout.

\newenvironment{compathscode}%
  {\par\noindent
   \advance\leftskip\mathindent
   \hscodestyle
   \let\\=\@normalcr
   \let\hspre\(\let\hspost\)%
   \pboxed}%
  {\endpboxed\)%
   \par\noindent
   \ignorespacesafterend}

\newcommand{\compaths}{\sethscode{compathscode}}

% "plain" mode is the proposed default.
% It should now work with \centering.
% This required some changes. The old version
% is still available for reference as oldplainhscode.

\newenvironment{plainhscode}%
  {\hsnewpar\abovedisplayskip
   \advance\leftskip\mathindent
   \hscodestyle
   \let\hspre\(\let\hspost\)%
   \pboxed}%
  {\endpboxed%
   \hsnewpar\belowdisplayskip
   \ignorespacesafterend}

\newenvironment{oldplainhscode}%
  {\hsnewpar\abovedisplayskip
   \advance\leftskip\mathindent
   \hscodestyle
   \let\\=\@normalcr
   \(\pboxed}%
  {\endpboxed\)%
   \hsnewpar\belowdisplayskip
   \ignorespacesafterend}

% Here, we make plainhscode the default environment.

\newcommand{\plainhs}{\sethscode{plainhscode}}
\newcommand{\oldplainhs}{\sethscode{oldplainhscode}}
\plainhs

% The arrayhscode is like plain, but makes use of polytable's
% parray environment which disallows page breaks in code blocks.

\newenvironment{arrayhscode}%
  {\hsnewpar\abovedisplayskip
   \advance\leftskip\mathindent
   \hscodestyle
   \let\\=\@normalcr
   \(\parray}%
  {\endparray\)%
   \hsnewpar\belowdisplayskip
   \ignorespacesafterend}

\newcommand{\arrayhs}{\sethscode{arrayhscode}}

% The mathhscode environment also makes use of polytable's parray 
% environment. It is supposed to be used only inside math mode 
% (I used it to typeset the type rules in my thesis).

\newenvironment{mathhscode}%
  {\parray}{\endparray}

\newcommand{\mathhs}{\sethscode{mathhscode}}

% texths is similar to mathhs, but works in text mode.

\newenvironment{texthscode}%
  {\(\parray}{\endparray\)}

\newcommand{\texths}{\sethscode{texthscode}}

% The framed environment places code in a framed box.

\def\codeframewidth{\arrayrulewidth}
\RequirePackage{calc}

\newenvironment{framedhscode}%
  {\parskip=\abovedisplayskip\par\noindent
   \hscodestyle
   \arrayrulewidth=\codeframewidth
   \tabular{@{}|p{\linewidth-2\arraycolsep-2\arrayrulewidth-2pt}|@{}}%
   \hline\framedhslinecorrect\\{-1.5ex}%
   \let\endoflinesave=\\
   \let\\=\@normalcr
   \(\pboxed}%
  {\endpboxed\)%
   \framedhslinecorrect\endoflinesave{.5ex}\hline
   \endtabular
   \parskip=\belowdisplayskip\par\noindent
   \ignorespacesafterend}

\newcommand{\framedhslinecorrect}[2]%
  {#1[#2]}

\newcommand{\framedhs}{\sethscode{framedhscode}}

% The inlinehscode environment is an experimental environment
% that can be used to typeset displayed code inline.

\newenvironment{inlinehscode}%
  {\(\def\column##1##2{}%
   \let\>\undefined\let\<\undefined\let\\\undefined
   \newcommand\>[1][]{}\newcommand\<[1][]{}\newcommand\\[1][]{}%
   \def\fromto##1##2##3{##3}%
   \def\nextline{}}{\) }%

\newcommand{\inlinehs}{\sethscode{inlinehscode}}

% The joincode environment is a separate environment that
% can be used to surround and thereby connect multiple code
% blocks.

\newenvironment{joincode}%
  {\let\orighscode=\hscode
   \let\origendhscode=\endhscode
   \def\endhscode{\def\hscode{\endgroup\def\@currenvir{hscode}\\}\begingroup}
   %\let\SaveRestoreHook=\empty
   %\let\ColumnHook=\empty
   %\let\resethooks=\empty
   \orighscode\def\hscode{\endgroup\def\@currenvir{hscode}}}%
  {\origendhscode
   \global\let\hscode=\orighscode
   \global\let\endhscode=\origendhscode}%

\makeatother
\EndFmtInput
%
%%format (bin (n) (k)) = "\Big(\vcenter{\xymatrix@R=1pt{" n "\\" k "}}\Big)"
%------------------------------------------------------------------------------%

%====== DEFINIR GRUPO E ELEMENTOS =============================================%

\group{G99}
\studentA{xxxxxx}{Nome }
\studentB{xxxxxx}{Nome }
\studentC{xxxxxx}{Nome }

%==============================================================================%

\begin{document}
\sffamily
\setlength{\parindent}{0em}
\emergencystretch 3em
\renewcommand{\baselinestretch}{1.25} 
\input{Cover}
\pagestyle{pagestyle}

\newgeometry{left=25mm,right=20mm,top=25mm,bottom=25mm}
\setlength{\parindent}{1em}

\section*{Preâmbulo}

Em \CP\ pretende-se ensinar a progra\-mação de computadores
como uma disciplina científica. Para isso parte-se de um repertório de \emph{combinadores}
que formam uma álgebra da programação % (conjunto de leis universais e seus corolários)
e usam-se esses combinadores para construir programas \emph{composicionalmente},
isto é, agregando programas já existentes.

Na sequência pedagógica dos planos de estudo dos cursos que têm
esta disciplina, opta-se pela aplicação deste método à programação
em \Haskell\ (sem prejuízo da sua aplicação a outras linguagens
funcionais). Assim, o presente trabalho prático coloca os
alunos perante problemas concretos que deverão ser implementados em
\Haskell. Há ainda um outro objectivo: o de ensinar a documentar
programas, a validá-los e a produzir textos técnico-científicos de
qualidade.

Antes de abordarem os problemas propostos no trabalho, os grupos devem ler
com atenção o anexo \ref{sec:documentacao} onde encontrarão as instruções
relativas ao \emph{software} a instalar, etc.

Valoriza-se a escrita de \emph{pouco} código que corresponda a soluções
simples e elegantes que utilizem os combinadores de ordem superior estudados
na disciplina.

\noindent \textbf{Avaliação}. Faz parte da avaliação do trabalho a sua defesa
por parte dos elementos de cada grupo. Estes devem estar preparados para
responder a perguntas sobre \emph{qualquer} dos problemas deste enunciado.
A prestação \emph{individual} de cada aluno nessa defesa oral será uma componente
importante e diferenciadora da avaliação.


\Problema
Uma serialização (ou travessia) de uma árvore é uma sua representação sob a forma de uma lista. 
Na biblioteca \ensuremath{{\BTree}} encontram-se as funções de serialização \ensuremath{\Varid{inordt}}, \ensuremath{\Varid{preordt}} e \ensuremath{\Varid{postordt}},
que fazem as travessias \emph{in-order}, \emph{ pre-order} e \emph{post-order}, respectivamente.
Todas essas travessias são catamorfismos que percorrem a árvore argumento em regime \emph{depth-first}.

Pretende-se agora uma função \ensuremath{\Varid{bforder}} que faça a travessia em regime \emph{breadth-first},
isto é, por níveis.
Por exemplo, para a árvore \ensuremath{t_1 } dada em anexo e mostrada na figura a seguir,

\begin{center}
	\figura
\end{center}

\noindent a função deverá dar a lista

\begin{hscode}\SaveRestoreHook
\column{B}{@{}>{\hspre}l<{\hspost}@{}}%
\column{9}{@{}>{\hspre}l<{\hspost}@{}}%
\column{E}{@{}>{\hspre}l<{\hspost}@{}}%
\>[9]{}[\mskip1.5mu \mathrm{5},\mathrm{3},\mathrm{7},\mathrm{1},\mathrm{4},\mathrm{6},\mathrm{8}\mskip1.5mu]{}\<[E]%
\ColumnHook
\end{hscode}\resethooks

\noindent em que se vê como os níveis \ensuremath{\mathrm{5}}, depois \ensuremath{\mathrm{3},\mathrm{7}} e finalmente \ensuremath{\mathrm{1},\mathrm{4},\mathrm{6},\mathrm{8}} foram percorridos.

Pretendemos propor duas versões dessa função:

\begin{enumerate}
\item	Uma delas envolve um catamorfismo de \ensuremath{{\BTree}}s:
\begin{hscode}\SaveRestoreHook
\column{B}{@{}>{\hspre}l<{\hspost}@{}}%
\column{E}{@{}>{\hspre}l<{\hspost}@{}}%
\>[B]{}\Varid{bfsLevels}\mathbin{::}{\BTree}\;\Varid{a}\to [\mskip1.5mu \Varid{a}\mskip1.5mu]{}\<[E]%
\\
\>[B]{}\Varid{bfsLevels}\mathrel{=}\Varid{concat}\comp \Varid{levels}{}\<[E]%
\ColumnHook
\end{hscode}\resethooks
Complete a definição desse catamorfismo:
\begin{hscode}\SaveRestoreHook
\column{B}{@{}>{\hspre}l<{\hspost}@{}}%
\column{E}{@{}>{\hspre}l<{\hspost}@{}}%
\>[B]{}\Varid{levels}\mathbin{::}{\BTree}\;\Varid{a}\to [\mskip1.5mu [\mskip1.5mu \Varid{a}\mskip1.5mu]\mskip1.5mu]{}\<[E]%
\\
\>[B]{}\Varid{levels}\mathrel{=}\llparenthesis\, \Varid{glevels}\,\rrparenthesis{}\<[E]%
\ColumnHook
\end{hscode}\resethooks
\item A segunda proposta,
\begin{hscode}\SaveRestoreHook
\column{B}{@{}>{\hspre}l<{\hspost}@{}}%
\column{E}{@{}>{\hspre}l<{\hspost}@{}}%
\>[B]{}\Varid{bft}\mathbin{::}{\BTree}\;\Varid{a}\to [\mskip1.5mu \Varid{a}\mskip1.5mu]{}\<[E]%
\ColumnHook
\end{hscode}\resethooks
deverá basear-se num anamorfismo de listas.
\end{enumerate}
\textbf{Sugestão}: estudar o artigo \cite{Ok00} cujo PDF está incluído no material deste trabalho. 
Quando fizer testes ao seu código pode, se desejar, usar funções disponíveis na biblioteca
\ensuremath{{\Exp}} para visualizar as árvores em \graphviz{GraphViz} (formato \emph{.dot}).

Justifique devidamente a sua resolução, que deverá vir acompanhada de diagramas explicativos.
Como já se disse, valoriza-se a escrita de \emph{pouco} código que corresponda a soluções
simples e elegantes que utilizem os combinadores de ordem superior estudados
na disciplina.

\Problema

Considere a seguinte função em Haskell:
\begin{quote}
\begin{hscode}\SaveRestoreHook
\column{B}{@{}>{\hspre}l<{\hspost}@{}}%
\column{10}{@{}>{\hspre}l<{\hspost}@{}}%
\column{11}{@{}>{\hspre}l<{\hspost}@{}}%
\column{23}{@{}>{\hspre}l<{\hspost}@{}}%
\column{32}{@{}>{\hspre}l<{\hspost}@{}}%
\column{39}{@{}>{\hspre}l<{\hspost}@{}}%
\column{46}{@{}>{\hspre}l<{\hspost}@{}}%
\column{52}{@{}>{\hspre}l<{\hspost}@{}}%
\column{E}{@{}>{\hspre}l<{\hspost}@{}}%
\>[B]{}\Varid{f}\;\Varid{x}\mathrel{=}\Varid{wrapper}\comp \Varid{worker}\;\mathbf{where}{}\<[E]%
\\
\>[B]{}\hsindent{10}{}\<[10]%
\>[10]{}\Varid{wrapper}\mathrel{=}\Varid{head}{}\<[E]%
\\
\>[B]{}\hsindent{10}{}\<[10]%
\>[10]{}\Varid{worker}\;\mathrm{0}\mathrel{=}\Varid{start}\;\Varid{x}{}\<[E]%
\\
\>[B]{}\hsindent{10}{}\<[10]%
\>[10]{}\Varid{worker}\;(\Varid{n}\mathbin{+}\mathrm{1})\mathrel{=}\Varid{loop}\;\Varid{x}\;(\Varid{worker}\;\Varid{n}){}\<[E]%
\\[\blanklineskip]%
\>[B]{}\Varid{loop}\;\Varid{x}\;{}\<[11]%
\>[11]{}[\mskip1.5mu \Varid{s},{}\<[23]%
\>[23]{}\Varid{h},{}\<[32]%
\>[32]{}\Varid{k},{}\<[39]%
\>[39]{}\Varid{j},{}\<[46]%
\>[46]{}\Varid{m}{}\<[52]%
\>[52]{}\mskip1.5mu]\mathrel{=}{}\<[E]%
\\
\>[11]{}[\mskip1.5mu \Varid{h}\mathbin{/}\Varid{k}\mathbin{+}\Varid{s},\Varid{x}\mathbin{\uparrow}\mathrm{2}\mathbin{*}\Varid{h},\Varid{k}\mathbin{*}\Varid{j},\Varid{j}\mathbin{+}\Varid{m},\Varid{m}\mathbin{+}\mathrm{8}\mskip1.5mu]{}\<[E]%
\\[\blanklineskip]%
\>[B]{}\Varid{start}\;\Varid{x}\mathrel{=}[\mskip1.5mu \Varid{x},{}\<[23]%
\>[23]{}\Varid{x}\mathbin{\uparrow}\mathrm{3},{}\<[32]%
\>[32]{}\mathrm{6},{}\<[39]%
\>[39]{}\mathrm{20},{}\<[46]%
\>[46]{}\mathrm{22}{}\<[52]%
\>[52]{}\mskip1.5mu]{}\<[E]%
\ColumnHook
\end{hscode}\resethooks
\end{quote}
Pode-se provar pela lei de recursividade mútua que \ensuremath{\Varid{f}\;\Varid{x}\;\Varid{n}} calcula o seno hiperbólico de \ensuremath{\Varid{x}},
\ensuremath{\Varid{sinh}\;\Varid{x}}, para \ensuremath{\Varid{n}} aproximações da sua série de Taylor. 
Faça a derivação da função dada a partir da referida série de Taylor, apresentando
todos os cálculos justificativos, tal como se faz para outras funções no
capítulo respectivo do texto base desta UC \cite{Ol98-24}.

\Problema

A propor na 2ª edição deste enunciado.

\Problema

A propor na 2ª edição deste enunciado.

\part*{Anexos}

\appendix

\section{Natureza do trabalho a realizar}
\label{sec:documentacao}
Este trabalho teórico-prático deve ser realizado por grupos de 3 alunos.
Os detalhes da avaliação (datas para submissão do relatório e sua defesa
oral) são os que forem publicados na \cp{página da disciplina} na \emph{internet}.

Recomenda-se uma abordagem participativa dos membros do grupo em \textbf{todos}
os exercícios do trabalho, para assim poderem responder a qualquer questão
colocada na \emph{defesa oral} do relatório.

Para cumprir de forma integrada os objectivos do trabalho vamos recorrer
a uma técnica de programa\-ção dita ``\litp{literária}'' \cite{Kn92}, cujo
princípio base é o seguinte:
%
\begin{quote}\em
        Um programa e a sua documentação devem coincidir.
\end{quote}
%
Por outras palavras, o \textbf{código fonte} e a \textbf{documentação} de um
programa deverão estar no mesmo ficheiro.

O ficheiro \texttt{cp2526t.pdf} que está a ler é já um exemplo de
\litp{programação literária}: foi gerado a partir do texto fonte
\texttt{cp2526t.lhs}\footnote{O sufixo `lhs' quer dizer
\emph{\lhaskell{literate Haskell}}.} que encontrará no \MaterialPedagogico\
desta disciplina des\-com\-pactando o ficheiro \texttt{cp2526t.zip}.

Como se mostra no esquema abaixo, de um único ficheiro (\ensuremath{\Varid{lhs}})
gera-se um PDF ou faz-se a interpretação do código \Haskell\ que ele inclui:

        \esquema

Vê-se assim que, para além do \GHCi, serão necessários os executáveis \PdfLatex\ e
\LhsToTeX. Para facilitar a instalação e evitar problemas de versões e
conflitos com sistemas operativos, é recomendado o uso do \Docker\ tal como
a seguir se descreve.

\section{Docker} \label{sec:docker}

Recomenda-se o uso do \container\ cuja imagem é gerada pelo \Docker\ a partir do ficheiro
\texttt{Dockerfile} que se encontra na diretoria que resulta de descompactar
\texttt{cp2526t.zip}. Este \container\ deverá ser usado na execução
do \GHCi\ e dos comandos relativos ao \Latex. (Ver também a \texttt{Makefile}
que é disponibilizada.)

Após \href{https://docs.docker.com/engine/install/}{instalar o Docker} e
descarregar o referido zip com o código fonte do trabalho,
basta executar os seguintes comandos:
\begin{Verbatim}[fontsize=\small]
    $ docker build -t cp2526t .
    $ docker run -v ${PWD}:/cp2526t -it cp2526t
\end{Verbatim}
\textbf{NB}: O objetivo é que o container\ seja usado \emph{apenas} 
para executar o \GHCi\ e os comandos relativos ao \Latex.
Deste modo, é criado um \textit{volume} (cf.\ a opção \texttt{-v \$\{PWD\}:/cp2526t}) 
que permite que a diretoria em que se encontra na sua máquina local 
e a diretoria \texttt{/cp2526t} no \container\ sejam partilhadas.

Pretende-se então que visualize/edite os ficheiros na sua máquina local e que
os compile no \container, executando:
\begin{Verbatim}[fontsize=\small]
    $ lhs2TeX cp2526t.lhs > cp2526t.tex
    $ pdflatex cp2526t
\end{Verbatim}
\LhsToTeX\ é o pre-processador que faz ``pretty printing'' de código Haskell
em \Latex\ e que faz parte já do \container. Alternativamente, basta executar
\begin{Verbatim}[fontsize=\small]
    $ make
\end{Verbatim}
para obter o mesmo efeito que acima.

Por outro lado, o mesmo ficheiro \texttt{cp2526t.lhs} é executável e contém
o ``kit'' básico, escrito em \Haskell, para realizar o trabalho. Basta executar
\begin{Verbatim}[fontsize=\small]
    $ ghci cp2526t.lhs
\end{Verbatim}

\noindent Abra o ficheiro \texttt{cp2526t.lhs} no seu editor de texto preferido
e verifique que assim é: todo o texto que se encontra dentro do ambiente
\begin{quote}\small\tt
\text{\ttfamily \char92{}begin\char123{}code\char125{}}
\\ ... \\
\text{\ttfamily \char92{}end\char123{}code\char125{}}
\end{quote}
é seleccionado pelo \GHCi\ para ser executado.

\section{Em que consiste o TP}

Em que consiste, então, o \emph{relatório} a que se referiu acima?
É a edição do texto que está a ser lido, preenchendo o anexo \ref{sec:resolucao}
com as respostas. O relatório deverá conter ainda a identificação dos membros
do grupo de trabalho, no local respectivo da folha de rosto.

Para gerar o PDF integral do relatório deve-se ainda correr os comando seguintes,
que actualizam a bibliografia (com \Bibtex) e o índice remissivo (com \Makeindex),
\begin{Verbatim}[fontsize=\small]
    $ bibtex cp2526t.aux
    $ makeindex cp2526t.idx
\end{Verbatim}
e recompilar o texto como acima se indicou. (Como já se disse, pode fazê-lo
correndo simplesmente \texttt{make} no \container.)

No anexo \ref{sec:codigo} disponibiliza-se algum código \Haskell\ relativo
aos problemas que são colocados. Esse anexo deverá ser consultado e analisado
à medida que isso for necessário.

Deve ser feito uso da \litp{programação literária} para documentar bem o código que se
desenvolver, em particular fazendo diagramas explicativos do que foi feito e
tal como se explica no anexo \ref{sec:diagramas} que se segue.

\section{Como exprimir cálculos e diagramas em LaTeX/lhs2TeX} \label{sec:diagramas}

Como primeiro exemplo, estudar o texto fonte (\lhstotex{lhs}) do que está a ler\footnote{
Procure e.g.\ por \texttt{"sec:diagramas"}.} onde se obtém o efeito seguinte:\footnote{Exemplos
tirados de \cite{Ol98-24}.}
\begin{eqnarray*}
\start
\ensuremath{\Varid{id}\mathrel{=}\conj{\Varid{f}}{\Varid{g}}}
\just\equiv{ universal property }
\ensuremath{\begin{lcbr}\p1\comp \Varid{id}\mathrel{=}\Varid{f}\\\p2\comp \Varid{id}\mathrel{=}\Varid{g}\end{lcbr}}
\just\equiv{ identity }
\ensuremath{\begin{lcbr}\p1\mathrel{=}\Varid{f}\\\p2\mathrel{=}\Varid{g}\end{lcbr}}
\qed
\end{eqnarray*}

Os diagramas podem ser produzidos recorrendo à \emph{package} \Xymatrix, por exemplo:
\begin{eqnarray*}
\xymatrix@C=2cm{
    \ensuremath{\N_0}
           \ar[d]_-{\ensuremath{\cataNat{\Varid{g}}}}
&
    \ensuremath{\mathrm{1}\mathbin{+}\N_0}
           \ar[d]^{\ensuremath{\Varid{id}\mathbin{+}\cataNat{\Varid{g}}}}
           \ar[l]_-{\ensuremath{\mathsf{in}}}
\\
     \ensuremath{\Conid{B}}
&
     \ensuremath{\mathrm{1}\mathbin{+}\Conid{B}}
           \ar[l]^-{\ensuremath{\Varid{g}}}
}
\end{eqnarray*}

\section{Código fornecido}\label{sec:codigo}

\subsection*{Problema 1}

Árvores exemplo:
\begin{hscode}\SaveRestoreHook
\column{B}{@{}>{\hspre}l<{\hspost}@{}}%
\column{3}{@{}>{\hspre}l<{\hspost}@{}}%
\column{4}{@{}>{\hspre}l<{\hspost}@{}}%
\column{5}{@{}>{\hspre}l<{\hspost}@{}}%
\column{12}{@{}>{\hspre}l<{\hspost}@{}}%
\column{E}{@{}>{\hspre}l<{\hspost}@{}}%
\>[B]{}t_1 \mathbin{::}{\BTree}\;\Conid{Int}{}\<[E]%
\\
\>[B]{}t_1 \mathrel{=}\Conid{Node}\;(\mathrm{5},(\Conid{Node}\;(\mathrm{3},(\Conid{Node}\;(\mathrm{1},(\Conid{Empty},\Conid{Empty})),\Conid{Node}\;(\mathrm{4},(\Conid{Empty},\Conid{Empty})))),{}\<[E]%
\\
\>[B]{}\hsindent{12}{}\<[12]%
\>[12]{}\Conid{Node}\;(\mathrm{7},(\Conid{Node}\;(\mathrm{6},(\Conid{Empty},\Conid{Empty})),\Conid{Node}\;(\mathrm{8},(\Conid{Empty},\Conid{Empty})))))){}\<[E]%
\\[\blanklineskip]%
\>[B]{}t_2 \mathbin{::}{\BTree}\;\Conid{Int}{}\<[E]%
\\
\>[B]{}t_2 \mathrel{=}{}\<[E]%
\\
\>[B]{}\hsindent{3}{}\<[3]%
\>[3]{}\Varid{node}\;\mathrm{1}\;{}\<[E]%
\\
\>[3]{}\hsindent{2}{}\<[5]%
\>[5]{}(\Varid{node}\;\mathrm{2}\;(\Varid{node}\;\mathrm{4}\;\Conid{Empty}\;\Conid{Empty})\;(\Varid{node}\;\mathrm{5}\;\Conid{Empty}\;\Conid{Empty}))\;{}\<[E]%
\\
\>[3]{}\hsindent{2}{}\<[5]%
\>[5]{}(\Varid{node}\;\mathrm{3}\;(\Varid{node}\;\mathrm{6}\;\Conid{Empty}\;\Conid{Empty})\;(\Varid{node}\;\mathrm{7}\;\Conid{Empty}\;\Conid{Empty})){}\<[E]%
\\[\blanklineskip]%
\>[B]{}t_3 \mathbin{::}{\BTree}\;\Conid{Char}{}\<[E]%
\\
\>[B]{}t_3 \mathrel{=}{}\<[E]%
\\
\>[B]{}\hsindent{3}{}\<[3]%
\>[3]{}\Varid{node}\;\text{\ttfamily 'A'}\;{}\<[E]%
\\
\>[3]{}\hsindent{2}{}\<[5]%
\>[5]{}(\Varid{node}\;\text{\ttfamily 'B'}\;(\Varid{node}\;\text{\ttfamily 'C'}\;(\Varid{node}\;\text{\ttfamily 'D'}\;\Conid{Empty}\;\Conid{Empty})\;\Conid{Empty})\;\Conid{Empty})\;{}\<[E]%
\\
\>[3]{}\hsindent{2}{}\<[5]%
\>[5]{}(\Varid{node}\;\text{\ttfamily 'E'}\;\Conid{Empty}\;\Conid{Empty}){}\<[E]%
\\[\blanklineskip]%
\>[B]{}t_4 \mathbin{::}{\BTree}\;\Conid{Char}{}\<[E]%
\\
\>[B]{}t_4 \mathrel{=}{}\<[E]%
\\
\>[B]{}\hsindent{3}{}\<[3]%
\>[3]{}\Varid{node}\;\text{\ttfamily 'A'}\;{}\<[E]%
\\
\>[3]{}\hsindent{2}{}\<[5]%
\>[5]{}(\Varid{node}\;\text{\ttfamily 'B'}\;(\Varid{node}\;\text{\ttfamily 'C'}\;(\Varid{node}\;\text{\ttfamily 'D'}\;\Conid{Empty}\;\Conid{Empty})\;\Conid{Empty})\;\Conid{Empty})\;{}\<[E]%
\\
\>[3]{}\hsindent{2}{}\<[5]%
\>[5]{}\Conid{Empty}{}\<[E]%
\\[\blanklineskip]%
\>[B]{}t_5 \mathbin{::}{\BTree}\;\Conid{Int}{}\<[E]%
\\
\>[B]{}t_5 \mathrel{=}{}\<[E]%
\\
\>[B]{}\hsindent{3}{}\<[3]%
\>[3]{}\Varid{node}\;\mathrm{1}\;{}\<[E]%
\\
\>[3]{}\hsindent{1}{}\<[4]%
\>[4]{}(\Varid{node}\;\mathrm{2}\;(\Varid{node}\;\mathrm{4}\;\Conid{Empty}\;\Conid{Empty})\;\Conid{Empty})\;{}\<[E]%
\\
\>[3]{}\hsindent{1}{}\<[4]%
\>[4]{}(\Varid{node}\;\mathrm{3}\;\Conid{Empty}\;(\Varid{node}\;\mathrm{5}\;(\Varid{node}\;\mathrm{6}\;\Conid{Empty}\;\Conid{Empty})\;\Conid{Empty})){}\<[E]%
\\[\blanklineskip]%
\>[B]{}\Varid{node}\;\Varid{a}\;\Varid{b}\;\Varid{c}\mathrel{=}\Conid{Node}\;(\Varid{a},(\Varid{b},\Varid{c})){}\<[E]%
\ColumnHook
\end{hscode}\resethooks

%----------------- Soluções dos alunos -----------------------------------------%

\section{Soluções dos alunos}\label{sec:resolucao}
Os alunos devem colocar neste anexo as suas soluções para os exercícios
propostos, de acordo com o ``layout'' que se fornece.
Não podem ser alterados os nomes ou tipos das funções dadas, mas pode ser
adicionado texto ao anexo, bem como diagramas e/ou outras funções auxiliares
que sejam necessárias.

\noindent
\textbf{Importante}: Não pode ser alterado o texto deste ficheiro fora deste anexo.

\subsection*{Problema 1}

\begin{hscode}\SaveRestoreHook
\column{B}{@{}>{\hspre}l<{\hspost}@{}}%
\column{3}{@{}>{\hspre}l<{\hspost}@{}}%
\column{5}{@{}>{\hspre}l<{\hspost}@{}}%
\column{7}{@{}>{\hspre}l<{\hspost}@{}}%
\column{9}{@{}>{\hspre}l<{\hspost}@{}}%
\column{11}{@{}>{\hspre}l<{\hspost}@{}}%
\column{15}{@{}>{\hspre}l<{\hspost}@{}}%
\column{19}{@{}>{\hspre}l<{\hspost}@{}}%
\column{21}{@{}>{\hspre}l<{\hspost}@{}}%
\column{29}{@{}>{\hspre}l<{\hspost}@{}}%
\column{30}{@{}>{\hspre}l<{\hspost}@{}}%
\column{34}{@{}>{\hspre}l<{\hspost}@{}}%
\column{36}{@{}>{\hspre}l<{\hspost}@{}}%
\column{37}{@{}>{\hspre}l<{\hspost}@{}}%
\column{45}{@{}>{\hspre}l<{\hspost}@{}}%
\column{47}{@{}>{\hspre}l<{\hspost}@{}}%
\column{49}{@{}>{\hspre}l<{\hspost}@{}}%
\column{61}{@{}>{\hspre}l<{\hspost}@{}}%
\column{62}{@{}>{\hspre}l<{\hspost}@{}}%
\column{E}{@{}>{\hspre}l<{\hspost}@{}}%
\>[B]{}\Varid{concatPointWise}\mathbin{::}[\mskip1.5mu [\mskip1.5mu \Varid{a}\mskip1.5mu]\mskip1.5mu]\to [\mskip1.5mu [\mskip1.5mu \Varid{a}\mskip1.5mu]\mskip1.5mu]\to [\mskip1.5mu [\mskip1.5mu \Varid{a}\mskip1.5mu]\mskip1.5mu]{}\<[E]%
\\
\>[B]{}\Varid{concatPointWise}\;[\mskip1.5mu \mskip1.5mu]\;\Varid{ys}\mathrel{=}\Varid{ys}{}\<[E]%
\\
\>[B]{}\Varid{concatPointWise}\;\Varid{xs}\;[\mskip1.5mu \mskip1.5mu]\mathrel{=}\Varid{xs}{}\<[E]%
\\
\>[B]{}\Varid{concatPointWise}\;(\Varid{x}\mathbin{:}\Varid{xs})\;(\Varid{y}\mathbin{:}\Varid{ys})\mathrel{=}(\Varid{x}\mathbin{+\!\!+}\Varid{y})\mathbin{:}\Varid{concatPointWise}\;\Varid{xs}\;\Varid{ys}{}\<[E]%
\\[\blanklineskip]%
\>[B]{}\Varid{concatPointFree}\mathbin{::}[\mskip1.5mu [\mskip1.5mu \Varid{a}\mskip1.5mu]\mskip1.5mu]\to [\mskip1.5mu [\mskip1.5mu \Varid{a}\mskip1.5mu]\mskip1.5mu]\to [\mskip1.5mu [\mskip1.5mu \Varid{a}\mskip1.5mu]\mskip1.5mu]{}\<[E]%
\\
\>[B]{}\Varid{concatPointFree}\mathrel{=}\overline{\anaList{\Varid{gene}}}{}\<[E]%
\\
\>[B]{}\hsindent{5}{}\<[5]%
\>[5]{}\mathbf{where}{}\<[E]%
\\
\>[5]{}\hsindent{4}{}\<[9]%
\>[9]{}\Varid{gene}\mathrel{=}\alt{\alt{\Varid{g1}}{\Varid{g2}}}{\alt{\Varid{g3}}{\Varid{g4}}}\comp (\Varid{distr}+\Varid{distr})\comp \Varid{distl}\comp (\Varid{outList}\times\Varid{outList}){}\<[E]%
\\
\>[5]{}\hsindent{4}{}\<[9]%
\>[9]{}\Varid{g1}\mathrel{=}i_1\comp \underline{}{}\<[E]%
\\
\>[5]{}\hsindent{4}{}\<[9]%
\>[9]{}\Varid{g2}\mathrel{=}i_2\comp (\Varid{id}\times\conj{\Varid{nil}}{\Varid{id}})\comp \p2{}\<[E]%
\\
\>[5]{}\hsindent{4}{}\<[9]%
\>[9]{}\Varid{g3}\mathrel{=}i_2\comp (\Varid{id}\times\conj{\Varid{id}}{\Varid{nil}})\comp \p1{}\<[E]%
\\
\>[5]{}\hsindent{4}{}\<[9]%
\>[9]{}\Varid{g4}\mathrel{=}i_2\comp \conj{\mathsf{conc}\comp (\p1\times\p1)}{\p2\times\p2}{}\<[E]%
\\[\blanklineskip]%
\>[B]{}\mbox{\onelinecomment  Definição da Condicional (dada no enunciado)}{}\<[E]%
\\
\>[B]{}\mbox{\onelinecomment  cond p f g = (either f g) . (grd p)}{}\<[E]%
\\
\>[B]{}\Varid{concatCondicional}\mathbin{::}[\mskip1.5mu [\mskip1.5mu \Varid{a}\mskip1.5mu]\mskip1.5mu]\to [\mskip1.5mu [\mskip1.5mu \Varid{a}\mskip1.5mu]\mskip1.5mu]\to [\mskip1.5mu [\mskip1.5mu \Varid{a}\mskip1.5mu]\mskip1.5mu]{}\<[E]%
\\
\>[B]{}\Varid{concatCondicional}\mathrel{=}\overline{\anaList{\Varid{gene}}}{}\<[E]%
\\
\>[B]{}\hsindent{5}{}\<[5]%
\>[5]{}\mathbf{where}{}\<[E]%
\\
\>[5]{}\hsindent{2}{}\<[7]%
\>[7]{}\mbox{\onelinecomment  O Gene: Uma cadeia de condicionais (If-Else-If...)}{}\<[E]%
\\
\>[5]{}\hsindent{2}{}\<[7]%
\>[7]{}\Varid{gene}\mathrel{=}\Varid{cond}\;\Varid{ambasVazias}\;{}\<[36]%
\>[36]{}\mbox{\onelinecomment  if both empty}{}\<[E]%
\\
\>[7]{}\hsindent{8}{}\<[15]%
\>[15]{}\Varid{stop}\;{}\<[37]%
\>[37]{}\mbox{\onelinecomment  stop}{}\<[E]%
\\
\>[7]{}\hsindent{8}{}\<[15]%
\>[15]{}(\Varid{cond}\;\Varid{listaEsquerdaVazia}\;{}\<[47]%
\>[47]{}\mbox{\onelinecomment  if left list}{}\<[E]%
\\
\>[15]{}\hsindent{4}{}\<[19]%
\>[19]{}\Varid{doListaDireita}\;{}\<[49]%
\>[49]{}\mbox{\onelinecomment  Se só L1 vazia -> Processa L2}{}\<[E]%
\\
\>[15]{}\hsindent{4}{}\<[19]%
\>[19]{}(\Varid{cond}\;\Varid{listaDireitaVazia}\;{}\<[47]%
\>[47]{}\mbox{\onelinecomment  if right list}{}\<[E]%
\\
\>[19]{}\hsindent{2}{}\<[21]%
\>[21]{}\Varid{doListaEsquerda}\;{}\<[45]%
\>[45]{}\mbox{\onelinecomment  Se só L2 vazia -> Processa L1}{}\<[E]%
\\
\>[19]{}\hsindent{2}{}\<[21]%
\>[21]{}\Varid{doAmbas})){}\<[34]%
\>[34]{}\mbox{\onelinecomment  Senão -> Processa ambas (caso x:xs, y:ys)}{}\<[E]%
\\[\blanklineskip]%
\>[5]{}\hsindent{2}{}\<[7]%
\>[7]{}\mbox{\onelinecomment  1. Predicados}{}\<[E]%
\\
\>[5]{}\hsindent{2}{}\<[7]%
\>[7]{}\Varid{ambasVazias}\mathrel{=}\uncurry{(\mathrel{\wedge})}\comp (\Varid{null}\times\Varid{null})\mbox{\onelinecomment  ([], [])?}{}\<[E]%
\\
\>[5]{}\hsindent{2}{}\<[7]%
\>[7]{}\Varid{listaEsquerdaVazia}{}\<[30]%
\>[30]{}\mathrel{=}\Varid{null}\comp \p1{}\<[62]%
\>[62]{}\mbox{\onelinecomment  ([], ...)?}{}\<[E]%
\\
\>[5]{}\hsindent{2}{}\<[7]%
\>[7]{}\Varid{listaDireitaVazia}{}\<[29]%
\>[29]{}\mathrel{=}\Varid{null}\comp \p2{}\<[61]%
\>[61]{}\mbox{\onelinecomment  (..., [])?}{}\<[E]%
\\[\blanklineskip]%
\>[5]{}\hsindent{2}{}\<[7]%
\>[7]{}\mbox{\onelinecomment  2. Ações (Os ramos)}{}\<[E]%
\\[\blanklineskip]%
\>[5]{}\hsindent{2}{}\<[7]%
\>[7]{}\mbox{\onelinecomment  Caso Paragem: Devolve Left ()}{}\<[E]%
\\
\>[5]{}\hsindent{2}{}\<[7]%
\>[7]{}\Varid{stop}\mathrel{=}i_1\comp \underline{}{}\<[E]%
\\[\blanklineskip]%
\>[5]{}\hsindent{2}{}\<[7]%
\>[7]{}\mbox{\onelinecomment  Caso L1 Vazia ([], y:ys): Produz y, estado ([], ys)}{}\<[E]%
\\
\>[5]{}\hsindent{2}{}\<[7]%
\>[7]{}\mbox{\onelinecomment  Nota: Usamos head/tail porque o predicado garante que não é vazia}{}\<[E]%
\\
\>[5]{}\hsindent{2}{}\<[7]%
\>[7]{}\Varid{doListaDireita}\mathrel{=}i_2\comp \conj{\Varid{head}\comp \p2}{\conj{\Varid{nil}}{\Varid{tail}\comp \p2}}{}\<[E]%
\\[\blanklineskip]%
\>[5]{}\hsindent{2}{}\<[7]%
\>[7]{}\mbox{\onelinecomment  Caso L2 Vazia (x:xs, []): Produz x, estado (xs, [])}{}\<[E]%
\\
\>[5]{}\hsindent{2}{}\<[7]%
\>[7]{}\Varid{doListaEsquerda}\mathrel{=}i_2\comp \conj{\Varid{head}\comp \p1}{\conj{\Varid{tail}\comp \p1}{\Varid{nil}}}{}\<[E]%
\\[\blanklineskip]%
\>[5]{}\hsindent{2}{}\<[7]%
\>[7]{}\mbox{\onelinecomment  Caso Ambas Cheias (x:xs, y:ys): Produz x++y, estado (xs, ys)}{}\<[E]%
\\
\>[5]{}\hsindent{2}{}\<[7]%
\>[7]{}\Varid{doAmbas}\mathrel{=}i_2\comp \conj{\mathsf{conc}\comp (\Varid{head}\times\Varid{head})}{\Varid{tail}\times\Varid{tail}}{}\<[E]%
\\[\blanklineskip]%
\>[B]{}\mbox{\onelinecomment  Define helper e uma árvore de teste (cola isto no GHCi)}{}\<[E]%
\\[\blanklineskip]%
\>[B]{}\Varid{glevelsPointWise}\mathbin{::}()+(\Varid{a},([\mskip1.5mu [\mskip1.5mu \Varid{a}\mskip1.5mu]\mskip1.5mu],[\mskip1.5mu [\mskip1.5mu \Varid{a}\mskip1.5mu]\mskip1.5mu]))\to [\mskip1.5mu [\mskip1.5mu \Varid{a}\mskip1.5mu]\mskip1.5mu]{}\<[E]%
\\
\>[B]{}\Varid{glevelsPointWise}\;(i_1\;())\mathrel{=}[\mskip1.5mu \mskip1.5mu]{}\<[E]%
\\
\>[B]{}\Varid{glevelsPointWise}\;(i_2\;(\Varid{a},(\Varid{ls},\Varid{rs})))\mathrel{=}[\mskip1.5mu \Varid{a}\mskip1.5mu]\mathbin{:}\Varid{concatPointWise}\;\Varid{ls}\;\Varid{rs}{}\<[E]%
\\[\blanklineskip]%
\>[B]{}\Varid{glevelsPointFree}\mathbin{::}()+(\Varid{a},([\mskip1.5mu [\mskip1.5mu \Varid{a}\mskip1.5mu]\mskip1.5mu],[\mskip1.5mu [\mskip1.5mu \Varid{a}\mskip1.5mu]\mskip1.5mu]))\to [\mskip1.5mu [\mskip1.5mu \Varid{a}\mskip1.5mu]\mskip1.5mu]{}\<[E]%
\\
\>[B]{}\Varid{glevelsPointFree}\mathrel{=}\alt{\Varid{nil}}{\Varid{cons}\comp (\Varid{singl}\times\uncurry{\Varid{concatPointFree}})}{}\<[E]%
\\[\blanklineskip]%
\>[B]{}\Varid{genePointFree}\mathbin{::}[\mskip1.5mu {\BTree}\;\Varid{a}\mskip1.5mu]\to ()+(\Varid{a},[\mskip1.5mu {\BTree}\;\Varid{a}\mskip1.5mu]){}\<[E]%
\\
\>[B]{}\Varid{genePointFree}\;[\mskip1.5mu \mskip1.5mu]\mathrel{=}i_1\;(){}\<[E]%
\\
\>[B]{}\Varid{genePointFree}\;(\Conid{Empty}\mathbin{:}\Varid{ts})\mathrel{=}\Varid{genePointFree}\;\Varid{ts}{}\<[E]%
\\
\>[B]{}\Varid{genePointFree}\;(\Conid{Node}\;(\Varid{a},(\Varid{l},\Varid{r}))\mathbin{:}\Varid{ts})\mathrel{=}i_2\;(\Varid{a},\Varid{ts}\mathbin{+\!\!+}[\mskip1.5mu \Varid{l},\Varid{r}\mskip1.5mu]){}\<[E]%
\\[\blanklineskip]%
\>[B]{}\Varid{genePointWise}\mathbin{::}[\mskip1.5mu {\BTree}\;\Varid{a}\mskip1.5mu]\to ()+(\Varid{a},[\mskip1.5mu {\BTree}\;\Varid{a}\mskip1.5mu]){}\<[E]%
\\
\>[B]{}\Varid{genePointWise}\;[\mskip1.5mu \mskip1.5mu]\mathrel{=}i_1\;(){}\<[E]%
\\
\>[B]{}\Varid{genePointWise}\;(\Conid{Empty}\mathbin{:}\Varid{ts})\mathrel{=}\Varid{genePointWise}\;\Varid{ts}{}\<[E]%
\\
\>[B]{}\Varid{genePointWise}\;(\Conid{Node}\;(\Varid{a},(\Varid{l},\Varid{r}))\mathbin{:}\Varid{ts})\mathrel{=}\Varid{gNode}\;(\Varid{a},(\Varid{ts},(\Varid{l},\Varid{r}))){}\<[E]%
\\
\>[B]{}\hsindent{3}{}\<[3]%
\>[3]{}\mathbf{where}{}\<[E]%
\\
\>[3]{}\hsindent{2}{}\<[5]%
\>[5]{}\Varid{gNode}\mathrel{=}i_2{}\<[E]%
\\
\>[5]{}\hsindent{6}{}\<[11]%
\>[11]{}\comp (\Varid{id}\times\mathsf{conc}){}\<[E]%
\\
\>[5]{}\hsindent{6}{}\<[11]%
\>[11]{}\comp (\Varid{id}\times(\Varid{id}\times\mathsf{conc})){}\<[E]%
\\
\>[5]{}\hsindent{6}{}\<[11]%
\>[11]{}\comp (\Varid{id}\times(\Varid{id}\times(\Varid{singl}\times\Varid{singl}))){}\<[E]%
\\
\>[B]{}\mbox{\onelinecomment  g1 . cons.(Node >< id).assocr.(id >< swap) = i2. (id >< conc) . (id >< (id >< conc)) . (id >< (id >< (singl >< singl)))}{}\<[E]%
\\[\blanklineskip]%
\>[B]{}\Varid{bft}\;\Varid{t}\mathrel{=}\anaList{\Varid{genePointWise}}\;(\Varid{singl}\;\Varid{t}){}\<[E]%
\ColumnHook
\end{hscode}\resethooks

\subsection*{Problema 2}

\subsection*{Problema 3}

\subsection*{Problema 4}

%----------------- Índice remissivo (exige makeindex) -------------------------%

\printindex

%----------------- Bibliografia (exige bibtex) --------------------------------%

\bibliographystyle{plain}
\bibliography{cp2526t}

%----------------- Fim do documento -------------------------------------------%
\end{document}
